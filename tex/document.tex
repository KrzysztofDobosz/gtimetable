\documentclass[a4paper,12pt]{article}
\usepackage{latexsym}
\usepackage[polish]{babel}
\usepackage[utf8]{inputenc}
\usepackage[OT4]{fontenc}
\usepackage{fancyhdr}
\usepackage{amsmath}
\usepackage{amsthm}
\usepackage[pdftex]{graphicx}
\usepackage{graphicx}
\fancypagestyle{plain}{
% zmiana liter w~żywej paginie na małe
%\renewcommand{\chaptermark}[1]{\markboth{#1}{}}
%\renewcommand{\sectionmark}[1]{\markright{\thesection\ #1}}
\fancyhf{} % usuń bieżące ustawienia pagin
\fancyhead[LE,RO]{\date{\today}}
\fancyhead[LO]{Lokalne Sieci Komputerowe, laboratorium}
\renewcommand{\headrulewidth}{1.0pt}
\renewcommand{\footrulewidth}{1pt}
\addtolength{\headheight}{1.5pt} % pionowy odstęp na kreskę
\fancyfoot[LE,CO]{Michał Brzeziński-Spiczak }
\fancyfoot[LE,RO]{\thepage}
\renewcommand{\headrulewidth}{1pt} % pozioma kreska
}
%\addtolength{\textheight}{3.0cm}
%\addtolength{\hoffset}{-1.3cm}
%\addtolength{\textwidth}{2.7cm}
%\addtolength{\marginparwidth}{-2cm}
%\voffset = -50pt


\begin{document}
\pagestyle{plain}

\begin{titlepage}
 


 
% Upper part of the page
\flushleft{\includegraphics[width=0.25\textwidth]{pwr}\\[1cm]
\textsc{Politechnika Wrocławska\\
Wydział Informatyki i Zarządzania\\
Kierunek: Informatyka\\
Rok: IV}}\\[1cm] 
 \begin{center}
%\textsc{\LARGE Lokalne Sieci Komputerowe\\Laboratorium}\\[1.5cm]
 
\textsc{ Sprawozdanie z ćwiczeń laboratoryjnych do przedmiotu LOKALNE
SIECI KOMPUTEROWE}\\[1.5cm]
 
 
% Title
 \hrule height 1pt
  \par
  \vfil \vskip 0.5cm
   
{ \large \bfseries GTT\\
GoogleMapsTimetable\\
Integracja rozkładu jazdy i Google Maps}\\[0.8cm]
 
  \hrule height 1pt
  \par
  \vfil \vskip 3.5cm

% Author and supervisor 
\begin{minipage}{0.55\textwidth}
\begin{flushleft} \large
\emph{Autorzy:}\\
Michał \textsc{Brzeziński-Spiczak}\\
\end{flushleft}
\end{minipage}
\begin{minipage}{0.4\textwidth}
\begin{flushright} \large
\emph{Prowadzący:} \\
dr inż. Ziemowit \textsc{Nowak}
\end{flushright}
\end{minipage}
 
\vfill
 
% Bottom of the page
{\large \today}
 
\end{center}
 
\end{titlepage}
\newpage
\tableofcontents
\newpage
%%%%%%właściwa treść:
\section*{Wprowadzenie} 


W miastach takich jaki Wrocław komunikacja miejska jest często najszybszą formą
transportu. Dla tych, którzy są w danym mieście przejazdem, w odpwiedzinach --
często też jedyną. Problemem może jednak okazać się wówczas nie sama prędkość
autobusów czy tramwajów, ale czas jaki trzeba poświęcić na odnalezienie odpowiedniej linii, połączenia.

Najprostsze wyjście -- przygotować się. Rozkłady
jazdy są ogólno dostępne. W Internecie pojawia się coraz więcej portali
oferujących wyszukiwarki połączeń. Jednak i w taki przypadku pojawia się
problem -- trzeba wiedzieć między jakimi przystankami znaleźć połączenie. Zatem
jaki przystanek jest najbliżej miejsca, do którego muszę się dostać?

\textbf{GTT \emph{GoogleMapsTimetable}} proponuje rozwiązanie i tego problemu:
integracja rozkładów jazdy komunikacji miejskiej, wyszukiwarki połączeń oraz
map (bazując na znanym systemie \emph{GoogleMaps}).

Wykorzystanie planu miasta wprowadza nowe możliwości -- w najogólniejszym
przypadku wystarczy wyklikac na mapie punkt startowy, następnie punkt docelowy
i wcisnąć \emph{Szukaj}, aby dowiedzieć się jak (i gdzie konkretnie, czyli na
jaki przystanek) dojechać, żeby znaleźć się tam, gdzie jest to konieczne.

Nie jest to z pewnością pomysł innowacyjny. Można odnaleźć ślady i wzmianki
po wielu systemach mających na celu podobne udogodnienie korzystania z
rozkładów jazdy komunikacji miejskiej, jednak działających systemów jest nie
wiele. W trakcie prac nad powstaniem \textbf{GTT} swojej premiery doczekał się
portal \emph{jakdojade.pl}\footnote{www.jakdojade.pl}, o którym można
powiedzieć, że w głównej mierze spełnia stawiane wymagania.

\section{Analiza wymagań funkcjonalnych}

\section{Zarys zasady działania aplikacji}
\begin{figure}[htp]
\centering
\includegraphics[width=0.7\textwidth]{pwr} 
\caption{Zarys działania aplikacji -- diagram}\label{dzialanie_systemu}
\end{figure}
\section{Analiza dostępnych danych wejściowych systemu}
Jako że głównym celem \textbf{GTT} jest integracja rozkładu jazdy, wyszukiwarki
połączeń oraz map konieczne jest zdefiniowanie i odpowiednie przetwarzanie
danych źródłowych. Wśród nich wymienić należy:
\begin{itemize}
  \item rozkład jazdy komunikacji miejskiej MPK Wrocław
  
  wstępna analiza wykazała dostępność rozkładów jazdy w postaci plików XML
  udostępnianych przez portal Urzędu Miasta Wrocław\footnote{http://www.wroclaw.pl/m3298/p3714.aspx}
  \item lokalizacja (współrzędne geograficzne przystanków)

   podobnie udostępniane przez portal Urzędu Miasta w postaci plików
   ShapeFile\footnote{http://www.wroclaw.pl/m3293/}
  \item mapy 
  
  ze względu na wstępne wymagania projektowe spośród dostępnych dostawców map
  internetowych wybrano \emph{GoogleMaps} jeden z serwisów firmy \emph{Google},
  który z założenia ma być bazą danych map oraz zdjęć satelitarnych i
  lotniczych całego świata, a ponadto do którego zgodnie z zasadami działa
  firmy istnieje co najmniej kilka API pozwalających na wykorzystanie map we
  własnych serwisach.
\end{itemize}

Więcej informacji na temat sposobu przetwarzania danych wejściowych w kolejnych
sekcjach dokumentacji.

\section{Struktura systemu}

Opisywana aplikacja charakteryzuje się klasyczną strukturą webową typu
klient-serwer, jaką przedstawia diagram \ref{schemat_klient_server}.

\begin{figure}[htp]
\centering
\includegraphics[width=0.7\textwidth]{pwr}
\caption{Ogólna struktura aplikacji}\label{schemat_klient_server}
\end{figure}

Na najwyższym poziomie ogólności przyjąć można, że część kliencka jest
interfejsem użytkownika: odpowiada za pobieranie danych wejściowych (zapytań do
systemu) oraz graficzną prezentację wyników działania systemu, natomiast część
serwerowa stanowi jądro realizujące zadania systemu.

Ze względu na fakt nałożenia znaczących wymagań funkcjonalnych odnośnie samego
interfejsu użytkownika część kliencka musi być traktowana na równi z serwerową.

\subsection{Część serwerowa}
Przyglądając się bliżej części serwerowej można wyodrębnić następujące moduły
funkcjonalne aplikacji:
\begin{itemize}
  \item zarządzanie bazą danych;
  \item zarządzanie pozyskiwaniem danych z zewnętrznych źródeł;
  \item silnik wyszukiwania połączeń;
  \item zarządzanie komunikacją z częścią kliencką
\end{itemize}
\subsubsection{Zarządzanie bazą danych}
Moduł zarządznia bazą danych stanowi łącznik między warstwą danych a
pozostałymi modułami serwera i pośrednio częścią kliencką. Jest najniższą
warstwą abstrakcji nałożoną na fizyczną realizację systemu zarządzania bazą
danych. 
 
\subsubsection{Pozyskiwaniem danych z zewnętrznych źródeł}
Moduł ten odpowiada za pobieranie i przetwarzanie danych z zewnętrznych źródeł
do postaci możliwej do zapisania w bazie danych.
 

\subsubsection{Silnik wyszukiwania połączeń}
Na podstawie relacyjnej struktury danych moduł ma za zadanie generowanie
najlepszych połączeń między zadanymi punktami.

\subsubsection{Komunikacja z częścią kliencką}

\subsection{Część kliencka}

Oddzielną kwestię stanowi komunikacja klient-serwer. W przypadku \textbf{GTT}
zdecydowano obarczyć odpowiedzialnością za komunikację framework \emph{GWT}, na
temat którego więcej informacji w sekcji Stosowanych rozwiązań technologicznych.
\subsection{Model obiektowy systemu}
\subsubsection{Diagram klas}
\section{Stosowane rozwiązania technologiczne}

\section{Raport z realizacji części serwerowej}
Prace nad częścią serwerową rozpoczęły się równolegle do przyswajania i
zaaklimatyzowania się z \emph{GWT Google Maps API}. Ustalona została logiczna
kolejność funkcjonalna tworzenia klas i realizacji modułów funkcjonalnych:
\begin{itemize}
  \item opracowanie struktury bazy danych
  \item analiza struktury dokumentów XML z rozkładami jazdy 
  \item programowe pozyskanie dokumentów XML z rozkładami
  \item wyodrębnienie interesujących danych -- parsowanie
  \item wypełnienie bazy danych
  \item opracowanie silnika wyszukiwania połączeń
  \item zdefiniowanie i oprogramowanie interfejsów pobierania informacji o
  rozkładach
  \item opracowanie interfejsów silnika wyszukiwania połączeń
  \item implementacja silnika wyszukiwania połączeń przystanek-przystanek
  \item analiza struktury i możliwości importu lokalizacji przystanków z plików
  źródłowych
  \item aktualizacja bazy danych o lokalizację przystanków
  \item implementacja silnika wyszukiwania połączęń punkt-punkt 
\end{itemize}
\section{Raport realizacji  części serwerowej} 
\section{Plany rozwojowe} 
\section{Podręcznik administratora}
\section{Podręcznik użytkownika}
\section{Literatura}


\end{document}