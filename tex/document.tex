\documentclass[a4paper,12pt]{article}
\usepackage{latexsym}
\usepackage[polish]{babel}
\usepackage[utf8]{inputenc}
\usepackage[OT4]{fontenc}
\usepackage{fancyhdr}
\usepackage{amsmath}
\usepackage{amsthm}
\usepackage[pdftex]{graphicx}
\usepackage{graphicx}
\fancypagestyle{plain}{
% zmiana liter w~żywej paginie na małe
%\renewcommand{\chaptermark}[1]{\markboth{#1}{}}
%\renewcommand{\sectionmark}[1]{\markright{\thesection\ #1}}
\fancyhf{} % usuń bieżące ustawienia pagin
\fancyhead[LE,RO]{\date{\today}}
\fancyhead[LO]{Lokalne Sieci Komputerowe, laboratorium}
\renewcommand{\headrulewidth}{1.0pt}
\renewcommand{\footrulewidth}{1pt}
\addtolength{\headheight}{1.5pt} % pionowy odstęp na kreskę
\fancyfoot[LE,CO]{Michał Brzeziński-Spiczak }
\fancyfoot[LE,RO]{\thepage}
\renewcommand{\headrulewidth}{1pt} % pozioma kreska
}
%\addtolength{\textheight}{3.0cm}
%\addtolength{\hoffset}{-1.3cm}
%\addtolength{\textwidth}{2.7cm}
%\addtolength{\marginparwidth}{-2cm}
%\voffset = -50pt


\begin{document}
\pagestyle{plain}

\begin{titlepage}
 


 
% Upper part of the page
\flushleft{\includegraphics[width=0.25\textwidth]{pwr}\\[1cm]
\textsc{Politechnika Wrocławska\\
Wydział Informatyki i Zarządzania\\
Kierunek: Informatyka\\
Rok: IV}}\\[1cm] 
 \begin{center}
%\textsc{\LARGE Lokalne Sieci Komputerowe\\Laboratorium}\\[1.5cm]
 
\textsc{ Sprawozdanie z ćwiczeń laboratoryjnych do przedmiotu LOKALNE
SIECI KOMPUTEROWE}\\[1.5cm]
 
 
% Title
 \hrule height 1pt
  \par
  \vfil \vskip 0.5cm
   
{ \large \bfseries GTT\\
GoogleMapsTimetable\\
Integracja rozkładu jazdy i Google Maps}\\[0.8cm]
 
  \hrule height 1pt
  \par
  \vfil \vskip 3.5cm

% Author and supervisor 
\begin{minipage}{0.55\textwidth}
\begin{flushleft} \large
\emph{Autorzy:}\\
Michał \textsc{Brzeziński-Spiczak}\\
\end{flushleft}
\end{minipage}
\begin{minipage}{0.4\textwidth}
\begin{flushright} \large
\emph{Prowadzący:} \\
dr inż. Ziemowit \textsc{Nowak}
\end{flushright}
\end{minipage}
 
\vfill
 
% Bottom of the page
{\large \today}
 
\end{center}
 
\end{titlepage}
\newpage
\tableofcontents
\newpage
%%%%%%właściwa treść:
\section*{Wprowadzenie} 


W miastach takich jaki Wrocław komunikacja miejska jest często najszybszą formą
transportu. Dla tych, którzy nie znają miasta często też jedyną. Problemem może
jednak okazać się wówczas nie sama prędkość autobusów czy tramwajów, ale czas
jaki trzeba poświęcić na odnalezienie odpowiedniej linii, połączenia. 

Najprostsze wyjście -- przygotować się. Rozkłady
jazdy są ogólno dostępne. W Internecie pojawia się coraz więcej portali
oferujących wyszukiwarki połączeń. Jednak i w taki przypadku pojawia się
problem -- trzeba wiedzieć między jakimi przystankami znaleźć połączenie. Zatem
jaki przystanek jest najbliżej miejsca, do którego muszę się dostać?

\textbf{GTT \emph{GoogleMapsTimetable}} proponuje rozwiązanie i tego problemu:
integracja rozkładów jazdy komunikacji miejskiej, wyszukiwarki połączeń oraz
map (bazując na znanym systemie \emph{GoogleMaps}).

Wykorzystanie planu miasta wprowadza nowe możliwości -- w najogólniejszym
przypadku wystarczy wyklikac na mapie punkt startowy, następnie punkt docelowy
i wcisnąć \emph{Szukaj}, aby dowiedzieć się jak (i gdzie konkretnie, czyli na
jaki przystanek) dojechać, żeby znaleźć się tam, gdzie jest to konieczne.
 

\section{Analiza wymagań funkcjonalnych}
\section{Stosowane rozwiązania technologiczne}
\section{Struktura systemu}
\section{Raport z realizacji części serwerowej}
\section{Raport realizacji  części serwerowej} 
\section{Plany rozwojowe} 
\section{Podręcznik administratora}
\section{Podręcznik użytkownika}
\section{Literatura}


\end{document}